\documentclass[conference]{IEEEtran}
\usepackage{xcolor}
\usepackage{tcolorbox}
\usepackage{graphicx}
\usepackage{fancyhdr} % Added for page numbering
\usepackage[colorlinks=true, linkcolor=blue, urlcolor=cyan, citecolor=cyan]{hyperref} % Added for colored 
\usepackage{enumitem} 
\usepackage{soul} 


\definecolor{deeppink}{rgb}{1.0, 0.08, 0.58}
\definecolor{awesome}{rgb}{1.0, 0.13, 0.32}
\definecolor{blue-green}{rgb}{0.0, 0.87, 0.87}
\definecolor{deeplilac}{rgb}{0.6, 0.33, 0.73}  
\definecolor{lightcarminepink}{rgb}{0.9, 0.4, 0.38}
\definecolor{ix}{rgb}{0.89, 0.15, 0.42}
\definecolor{parisgreen}{rgb}{0.31, 0.78, 0.47}
\definecolor{mauve}{rgb}{0.88, 0.69, 1.0}
\definecolor{lightpastelpurple}{rgb}{0.69, 0.61, 0.85}
\definecolor{ashgrey}{rgb}{0.7, 0.75, 0.71}
    
\title{DGL-25 Kaggle Competition: \emph{\textcolor{blue-green}{Your Team Name}, \textcolor{mauve}{Final Rank}}}
\author{\IEEEauthorblockN{Your Name 1, Student CID 1, Username 1} \\
\IEEEauthorblockN{Your Name 2, Student CID 2, Username 2} \\
\IEEEauthorblockN{Your Name 3, Student CID 3, Username 3} \\
\IEEEauthorblockN{Your Name 4, Student CID 4, Username 4} \\
\IEEEauthorblockN{Your Name 5, Student CID 5, Username 5} \\
} 

\begin{document}

\maketitle


%This report is worth \textcolor{deeppink}{75 points} of the total mark of the DGL projet.

\textbf{\textcolor{deeppink}{Important note:}} Remove all \textcolor{ashgrey}{Todo} markers and enter your responses. Do not remove the \textcolor{blue}{(points)} specifications for each question or sub-question.
 
%%%%%%%% ======= %%%%%%%% ======= %%%%%%%% ======= %%%%%%%% ======= %%%%%%%%
\section{Methodology \& Novelty \textcolor{blue}{(40 points)}}
%%%%%%%% ======= %%%%%%%% ======= %%%%%%%% ======= %%%%%%%% ======= %%%%%%%%

%%%%%%%%%%%%%%%%%%%%%%%%%%%%%%%%%%%%%%%%%%%
\subsection{Problem description \& motivation \textcolor{blue}{(5 points)}}
%%%%%%%%%%%%%%%%%%%%%%%%%%%%%%%%%%%%%%%%%%%

\textcolor{ashgrey}{Todo}.


%%%%%%%%%%%%%%%%%%%%%%%%%%%%%%%%%%%%%%%%%%%
\subsection{State-of-the art methods \textcolor{blue}{(3 points)}}
%%%%%%%%%%%%%%%%%%%%%%%%%%%%%%%%%%%%%%%%%%%

\textcolor{ashgrey}{Todo}.

\begin{table}[ht]
\renewcommand{\arraystretch}{1.3}
\centering
\caption{Overview of GNN-based SOTA Models}
\begin{tabular}{|p{2cm}|p{6cm}|}
\hline
\textbf{Model Name} & \textbf{Brief Description} \\
\hline
GraphSAGE \cite{graphsage} & Refines GCN by employing neighborhood sampling and aggregation techniques to efficiently learn node embeddings in large graphs. \\
\hline

Name \cite{} & Describe. \\
\hline

%Graph Attention Network (GAT) & Incorporates attention mechanisms to assign different importance levels to neighboring nodes. \\
%\hline

\end{tabular}
\label{tab:gnn_models}
\end{table}


%%%%%%%%%%%%%%%%%%%%%%%%%%%%%%%%%%%%%%%%%%%
\subsection{Main figure \textcolor{blue}{(4 points)}}
%%%%%%%%%%%%%%%%%%%%%%%%%%%%%%%%%%%%%%%%%%%

\textcolor{ashgrey}{Todo}.

%%%%%%%%%%%%%%%%%%%%%%%%%%%%%%%%%%%%%%%%%%%
\subsection{Brief overview of the proposed GNN \textcolor{blue}{(5 points)}}
%%%%%%%%%%%%%%%%%%%%%%%%%%%%%%%%%%%%%%%%%%%

\textcolor{ashgrey}{Todo}.

%%%%%%%%%%%%%%%%%%%%%%%%%%%%%%%%%%%%%%%%%%%
\subsection{Innovative components \textcolor{blue}{(10 points)}}
%%%%%%%%%%%%%%%%%%%%%%%%%%%%%%%%%%%%%%%%%%%

\textcolor{ashgrey}{Todo}.

\begin{table}[ht!]
\renewcommand{\arraystretch}{1.3}
\centering
\caption{Innovative Components of the Proposed GNN Framework}
\begin{tabular}{|p{3cm}|p{5cm}|}
\hline
\textbf{Novel Contribution} & \textbf{Rationale} \\
\hline
Contribution 1 & Justify the rationale behind this contribution. \\
\hline
Contribution 2 & Justify the rationale behind this contribution. \\
\hline
\end{tabular}
\label{tab:contributions}
\end{table}



%%%%%%%%%%%%%%%%%%%%%%%%%%%%%%%%%%%%%%%%%%%
\subsection{Mathematical properties of the proposed GNN \textcolor{blue}{(13 points)}}
%%%%%%%%%%%%%%%%%%%%%%%%%%%%%%%%%%%%%%%%%%%

\textcolor{ashgrey}{Todo}.

\begin{tcolorbox}[colframe=pink, colback=pink!20, title=Permutation invariance \textcolor{blue}{(5 points)}, height=4cm]
\begin{enumerate}[label=\alph*)]
\item \textcolor{ashgrey}{Todo}. \textcolor{blue}{(1 point)}
\item \textcolor{ashgrey}{Todo}. \textcolor{blue}{(4 points)}
\end{enumerate} 
 
\end{tcolorbox}

%\vspace{0.4cm}

\begin{tcolorbox}[colframe=lightcarminepink, colback=lightcarminepink!5, title=Permutation equivariance \textcolor{blue}{(5 points)}, height=4cm]
%[colframe=red, colback=red!10, title=Challenges or Research Questions, height=4cm]
\begin{enumerate}[label=\alph*)]
\item \textcolor{ashgrey}{Todo}. \textcolor{blue}{(1 point)}
\item \textcolor{ashgrey}{Todo}. \textcolor{blue}{(4 points)}
\end{enumerate}

\end{tcolorbox}


\begin{tcolorbox}[colframe=ix, colback=ix!5, title=Expressiveness \textcolor{blue}{(3 points)}, height=4cm]
\begin{enumerate}[label=\alph*)]
\item \textcolor{ashgrey}{Todo}. \textcolor{blue}{(1 point)}
\item \textcolor{ashgrey}{Todo}. \textcolor{blue}{(2 points)}
\end{enumerate}
\end{tcolorbox}


\vspace{1cm}

%%%%%%%% ======= %%%%%%%% ======= %%%%%%%% ======= %%%%%%%% ======= %%%%%%%%
\section{Experimental Setup \& Evaluation \textcolor{blue}{(27 points)}}
%%%%%%%% ======= %%%%%%%% ======= %%%%%%%% ======= %%%%%%%% ======= %%%%%%%%

%%%%%%%%%%%%%%%%%%%%%%%%%%%%%%%%%%%%%%%%%%%
\subsection{Results \textcolor{blue}{(9 points)}}
%%%%%%%%%%%%%%%%%%%%%%%%%%%%%%%%%%%%%%%%%%%

\begin{enumerate}[label=\alph*)]
    \item \textcolor{ashgrey}{Todo}. \textcolor{blue}{(2 points)}
    
    \item \textcolor{ashgrey}{Todo}. \textcolor{blue}{(4 points)}
    
    \item \textcolor{ashgrey}{Todo}. \textcolor{blue}{(2 points)}
    
    \item \textcolor{ashgrey}{Todo}. \textcolor{blue}{(1 point)}
\end{enumerate}

\begin{table}[ht!]
\renewcommand{\arraystretch}{1.3}
\centering
\caption{Additional Topological/Geometric Measures}
\begin{tabular}{|p{3cm}|p{4cm}|}
\hline
\textbf{Measure Name} & \textbf{Brief Description \& Rationale} \\
\hline
Measure 1 & Describe and justify. \\
\hline
Measure 2 & Describe and justify. \\
\hline
\end{tabular}
\label{tab:2Measures}
\end{table}



%%%%%%%%%%%%%%%%%%%%%%%%%%%%%%%%%%%%%%%%%%%
\subsection{Comparison Against Other Methods \textcolor{blue}{(6 points)}}
%%%%%%%%%%%%%%%%%%%%%%%%%%%%%%%%%%%%%%%%%%%

\textcolor{ashgrey}{Todo}.


%%%%%%%%%%%%%%%%%%%%%%%%%%%%%%%%%%%%%%%%%%%
\subsection{Scalability of Your Proposed GNN Model \textcolor{blue}{(7 points)}}
%%%%%%%%%%%%%%%%%%%%%%%%%%%%%%%%%%%%%%%%%%%

\textcolor{ashgrey}{Todo}.

%\begin{enumerate}[label=\alph*)]
   % \item How does your model's performance change as dataset size increases? % Provide empirical evidence or theoretical reasoning.
 %   \item Discuss memory and computational overhead. How does your architecture handle large-scale graphs? Does it require significant resources, or is it optimized for efficiency? Justify your reasoning.
  %  \item Suggest potential improvements to enhance scalability further.
%\end{enumerate}

%%%%%%%%%%%%%%%%%%%%%%%%%%%%%%%%%%%%%%%%%%%
\subsection{Reproducibility of Your Proposed GNN Model \textcolor{blue}{(5 points)}}
%%%%%%%%%%%%%%%%%%%%%%%%%%%%%%%%%%%%%%%%%%%

\textcolor{ashgrey}{Todo}.


%%%%%%%% ======= %%%%%%%% ======= %%%%%%%% ======= %%%%%%%% ======= %%%%%%%%
\section{Discussion \& Reflections \textcolor{blue}{(8 points)}}
%%%%%%%% ======= %%%%%%%% ======= %%%%%%%% ======= %%%%%%%% ======= %%%%%%%%

\begin{enumerate}[label=\alph*)]
\item \textcolor{ashgrey}{Todo}. \textcolor{blue}{(4 points)}

\begin{table}[ht!]
\renewcommand{\arraystretch}{1.3}
\centering
\caption{Strengths \& Weaknesses of the Proposed GNN}
\begin{tabular}{|p{2.8cm}|p{5cm}|}
\hline
Strength 1 & Provide an example. \\
\hline
Strength 2 & Provide an example. \\
\hline
Weakness 1 & Provide an example. \\
\hline
Weakness 2 & Provide an example. \\
\hline
\end{tabular}
\label{tab:reflections}
\end{table}

\item \textcolor{ashgrey}{Todo}. \textcolor{blue}{(4 points)}

\end{enumerate}


%%%%%%%%%%%%%%%%%%%%%%%%%%%%%%%%%%%%%%%%%%%%%%%%%%%%%%%%%%%%%%%%%%%%%%%%%%%%%%%%%%%%%%
%%%%%%%%%%%%%%%%%%%%%%%%%%%%%%%%%%%%%%%%%%%%%%%%%%%%%%%%%%%%%%%%%%%%%%%%%%%%%%%%%%%%%%

\thispagestyle{fancy} % Custom page style for page numbering
\fancyhf{} % Clear all headers and footers
\fancyfoot[C]{\thepage} % Center the page number in the footer

\bibliographystyle{IEEEtran}
\bibliography{references}

\end{document}
